% Created 2022-01-03 Mon 21:28
% Intended LaTeX compiler: pdflatex
\documentclass[11pt]{article}
\usepackage[utf8]{inputenc}
\usepackage{lmodern}
\usepackage[T1]{fontenc}
\usepackage[top=1in, bottom=1.in, left=1in, right=1in]{geometry}
\usepackage{graphicx}
\usepackage{longtable}
\usepackage{float}
\usepackage{wrapfig}
\usepackage{rotating}
\usepackage[normalem]{ulem}
\usepackage{amsmath}
\usepackage{textcomp}
\usepackage{marvosym}
\usepackage{wasysym}
\usepackage{amssymb}
\usepackage{amsmath}
\usepackage[theorems, skins]{tcolorbox}
\usepackage[version=3]{mhchem}
\usepackage[numbers,super,sort&compress]{natbib}
\usepackage{natmove}
\usepackage{url}
\usepackage[cache=false]{minted}
\usepackage[strings]{underscore}
\usepackage[linktocpage,pdfstartview=FitH,colorlinks,
linkcolor=blue,anchorcolor=blue,
citecolor=blue,filecolor=blue,menucolor=blue,urlcolor=blue]{hyperref}
\usepackage{attachfile}
\usepackage{setspace}
\usepackage[left=1in, right=1in, top=1in, bottom=1in, nohead]{geometry}
\geometry{margin=1.0in}
\usepackage{amsmath}
\usepackage{parskip}
\usepackage{graphicx}
\usepackage{framed,color}
\usepackage{epstopdf}
\usepackage{fancyhdr}
\usepackage{hyperref}
\usepackage[labelfont=bf]{caption}
\usepackage{setspace}
\setlength{\headheight}{10.2pt}
\setlength{\headsep}{20pt}
\def\dbar{{\mathchar'26\mkern-12mu d}}
\pagestyle{fancy}
\fancyhf{}
\renewcommand{\headrulewidth}{0.5pt}
\renewcommand{\footrulewidth}{0.5pt}
\lfoot{\today}
\cfoot{\copyright\ 2022 W.\ F.\ Schneider}
\rfoot{\thepage}
\chead{\bf{Career Choices for Chemical Engineers (CBE 20290)\vspace{12pt}}}
\lhead{\bf{Career Panels}}
\rhead{\bf{January 17, 2022}}
\usepackage{titlesec}
\titlespacing*{\section}
{0pt}{0.6\baselineskip}{0.2\baselineskip}
\title{University of Notre Dame\\Career Choices\\(CBE 20290)}
\author{Prof. William F.\ Schneider}
\def\dbar{{\mathchar'26\mkern-12mu d}}
\usepackage{siunitx}
\setcounter{secnumdepth}{3}
\author{William F. Schneider}
\date{\today}
\title{CBE 20290 Career Panel}
\begin{document}

\begin{OPTIONS}
\end{OPTIONS}

\section{Organization}
\label{sec:org48b26d9}
The weekly alumni career panels are intended to expose you to the diverse career paths taken by chemical engineers and  the diverse fields in which their skills can be applied in. They also provide an opportunity to practice your presentation and interviewing skills through interactions with alumni and current professionals.

Each session, one group of 3-4 students will be responsible for managing the panel discussion, including introductions, questions and answers, and written summary. The topical areas, panelists, and your assigned groups  are available on a \href{https://docs.google.com/spreadsheets/d/1cjuvP2S-zgWGZjiKhLU1Qg3ROS5dn2VucOxKP6T6Hpw/edit?usp=sharing}{Google sheet}  accessible from your ND account.  The Google sheet contains the email addresses and LinkedIn links of all the panelists.

\section{Panel agenda and tasks:}
\label{sec:org23820c3}
\begin{enumerate}
\item Introduce yourselves
\item Present a brief (1-2  minute, 1-2 slides) summary of the topical area
\item Present a brief (30 seconds, 1 slide) introduction of each panelist
\item Interview and moderate discussion with the panelists
\item Write a brief summary of the panel discussion
\end{enumerate}

To do this well will take some preparation. I suggest the team meet to set some rolls and responsibilities for the Introduction and final summary. Agree on individual roles:
\begin{itemize}
\item Background on the field
\item Introduce speaker 1
\item Introduce speaker 2
\item Introduce speaker 3
\item Timer/wrap-up/thank you
\end{itemize}

And work together on group roles:
\begin{itemize}
\item Questions/interview format
\item Scribe/summary
\item Post-panel thank you to panelists
\end{itemize}

You will want to contact the panelists ahead of time to obtain some biographical information, and the group will need to do research to describe the nature of the field (what services/produced does it provide, what is the size of the area, who are major players).

The group will also want to prepare a set of questions, or script, to guide the discussion with the speakers. You can draw from the \href{https://docs.google.com/document/d/1eUqnfeW1NTqqTqEzBxjCo6u20dFwN1kRAMNXPG9ioL0/edit?usp=sharing}{questions you prepared in Assignment 2}. The panel time is short, so think carefully about the topics you would like to cover and who you will ask the questions to. Use good interviewing skills. Each group member should  contribute to the interviewing. 

The group will take notes during the panel discussion and use them to annotate the prepared questions into a summary.

\section{Logistics}
\label{sec:org7a098ed}
Place your introductory slides in the class \href{https://drive.google.com/drive/folders/12p1B5icXV4FetwMoPTR7hkxTTPMj53qA?usp=sharing}{Google folder} by 9 am Friday of your panel session. Schneider will share those slides during the panel. Place your summary into the same folder by 8 am Monday following your session. 

\section{Graciousness}
\label{sec:org11f9ac3}
Remember that the alumni are knowledgeable and eager to help you. Remember too that they are taking valuable time out of their day. Be well prepared, dress appropriately, be courteous, and be grateful. 
\end{document}