% Created 2022-01-03 Mon 21:56
% Intended LaTeX compiler: pdflatex
\documentclass[11pt]{article}
\usepackage[utf8]{inputenc}
\usepackage{lmodern}
\usepackage[T1]{fontenc}
\usepackage[top=1in, bottom=1.in, left=1in, right=1in]{geometry}
\usepackage{graphicx}
\usepackage{longtable}
\usepackage{float}
\usepackage{wrapfig}
\usepackage{rotating}
\usepackage[normalem]{ulem}
\usepackage{amsmath}
\usepackage{textcomp}
\usepackage{marvosym}
\usepackage{wasysym}
\usepackage{amssymb}
\usepackage{amsmath}
\usepackage[theorems, skins]{tcolorbox}
\usepackage[version=3]{mhchem}
\usepackage[numbers,super,sort&compress]{natbib}
\usepackage{natmove}
\usepackage{url}
\usepackage[cache=false]{minted}
\usepackage[strings]{underscore}
\usepackage[linktocpage,pdfstartview=FitH,colorlinks,
linkcolor=blue,anchorcolor=blue,
citecolor=blue,filecolor=blue,menucolor=blue,urlcolor=blue]{hyperref}
\usepackage{attachfile}
\usepackage{setspace}
\usepackage[left=1in, right=1in, top=1in, bottom=1in, nohead]{geometry}
\geometry{margin=1.0in}
\usepackage{amsmath}
\usepackage{graphicx}
\usepackage{epstopdf}
\usepackage{fancyhdr}
\usepackage{hyperref}
\usepackage[labelfont=bf]{caption}
\usepackage{setspace}
\def\dbar{{\mathchar'26\mkern-12mu d}}
\pagestyle{fancy}
\fancyhf{}
\renewcommand{\headrulewidth}{0.5pt}
\renewcommand{\footrulewidth}{0.5pt}
\lfoot{\today}
\cfoot{\copyright\ 2021 W.\ F.\ Schneider}
\rfoot{\thepage}
\title{University of Notre Dame\\Physical Chemistry for Chemical Engineers\\(CHE 30324)}
\author{Prof. William F.\ Schneider}
\def\dbar{{\mathchar'26\mkern-12mu d}}
\usepackage[small]{titlesec}
\titlespacing*{\section}
{0pt}{0.4\baselineskip}{0.0\baselineskip}
\titlespacing*{\subsection}
{0pt}{0.4\baselineskip}{0.0\baselineskip}
\titlespacing*{\subsubsection}
{0pt}{0.1\baselineskip}{0.0\baselineskip}
\setcounter{secnumdepth}{3}
\author{William F. Schneider}
\date{\today}
\title{CBE 20290  Syllabus}
\begin{document}

\begin{OPTIONS}
\end{OPTIONS}

\begin{center}
\textsc{Career Choices for Chemical Engineers (CBE 20290)}

University of Notre Dame, Spring 2022
\end{center}

\begin{center}
\begin{tabular}{lr}
\hline
Prof.~Bill Schneider & \href{https://www.nd.edu/\~wschnei1}{web}\\
wschneider@nd.edu, phone 574-631-8754 & \href{https://www.linkedin.com/in/william-schneider-570091a/}{LinkedIn}\\
DBRT 136, F 3:30-4:20 & \href{https://notredame.zoom.us/meeting/96943641641}{Zoom link}\\
\hline
\end{tabular}
\end{center}

\section{Your Career Choices}
\label{sec:orgcad128e}
As you have undoubtedly already figured out, Chemical Engineering is a demanding major. Your courses will prepare you to solve complex technical problems of importance to society, to do so ethically, to work in teams, and to communicate effectively.  Those skills can take your career in an almost unlimited number of directions. This course is designed to expose you to some of those many options and to provide some of the professional skills needed to pursue them. Specifically, the learning goals of this 1 credit course are:

\begin{enumerate}
\item to be exposed to chemical engineering career options, through interactions with alumni who have preceeded you in this journey
\item to begin to discern options that are appealling to you
\item to develop  professional networking and presentation skills that will help get you where you want to go
\end{enumerate}

\section{Format}
\label{sec:org2809919}
The course will be conducted in person in 136 DBRT.  Weeks 1-2 will focus on professional skill development. Weeks 3-14 will involve a series of virtual panel discussions with alumni of the department. Each panel will focus on one sector, such as pharmaceuticals, energy, foods, personal care, \ldots. A group of students (you!) will introduce the topic and interview the visitors. Each of you will have a chance to participate in one of the panels.

\noindent Shared course materials available on \href{https://drive.google.com/drive/folders/12p1B5icXV4FetwMoPTR7hkxTTPMj53qA?usp=sharing}{Google Drive} and on \href{https://github.com/wmfschneider/CBE20290}{GitHub}.

\section{Schedule}
\label{sec:orgb7e623f}
\begin{center}
\begin{tabular}{lll}
January 14 & Introduction/Resumes/Handshake & \href{https://www.linkedin.com/in/loriann-edinborough/}{LoriAnn Edinborough}, Career Center\\
January 21 & Interviewing & \href{https://www.linkedin.com/in/loriann-edinborough/}{LoriAnn Edinborough}, Career Center\\
January 28 & Alumni Panel: Pharmaceuticals & student interviewer team\\
February 4 & Alumni Panel: Chemicals & student interviewer team\\
February 11 & Alumni Panel: Info Tech & student interviewer team\\
February 18 & Alumni Panel: Consumer Goods & student interviewer team\\
February 25 & Alumni Panel: Professional/Grad School & student interviewer team\\
March 2 & Alumni Panel: Oil \& Gas & \\
March 10 & Spring Break & \\
March 18 & Alumni Panel: Foods & student interviewer team\\
March 25 & Alumni Panel: Entrepreneurship & student interviewer team\\
April 1 & Alumni Panel: Energy/Renewables & student interviewer team\\
April 8 & Alumni Panel: Biotech & student interviewer team\\
April 15 & Good Friday & \\
April 22 & Alumni Panel: Consulting & student interviewer team\\
\end{tabular}
\end{center}

\href{https://calendar.google.com/calendar/u/0?cid=Y183NG02cDJnYWQ2NDQ4OTUzZGthaHJia2Nnc0Bncm91cC5jYWxlbmRhci5nb29nbGUuY29t}{Google course calendar}      \href{https://notredame.hosted.panopto.com/Panopto/Pages/Sessions/List.aspx?folderID=b93d46fc-da0a-4736-b7e1-acc1018685ee}{Panopto recordings} 

\section{Assignments}
\label{sec:org079724f}
\begin{center}
\begin{tabular}{lll}
Due January 10 & \href{https://forms.gle/TsqC2PDdJU14JwN49}{Career Reflection} & \\
Due January 17 & \href{https://forms.gle/yRDjwjk5mkxFGZTt8}{Interview questions} & \\
Rolling basis & \href{./Assignments/Interview.org}{Career Panel} & \href{./Assignments/Interview.pdf}{pdf}\\
Due April 29 & \href{https://forms.gle/GjdNjfbhfEXnvLEF6}{Final Reflection} & \\
\end{tabular}
\end{center}

\section{Grading}
\label{sec:org63b1778}
The course is graded S/U. To receive an S, you must complete all assignments.

\section{Mental Health}
\label{sec:org0b29cee}
Care and Wellness Consultants provide support and resources to students who are experiencing stressful or difficult situations that may be interfering with academic progress. Through Care and Wellness Consultants, students can be referred to The University Counseling
Center (for cost-free and confidential psychological and psychiatric services from
licensed professionals), University Health Services (which provides primary care,
psychiatric services, case management, and a pharmacy), and The McDonald
Center for Student Well Being (for problems with sleep, stress, and substance
use). Visit \url{https://care.nd.edu}.

\section{Professional courtesy}
\label{sec:org513749d}
\begin{enumerate}
\item Arrive at each session on time, early if possible.
\item Turn your camera on. Make sure you are sitting up and acting professionally.
\item Dress appropriately.
\item Research the organization, the presenters’ backgrounds, or industries. Be prepared!
\item You can ask questions through the chat or by unmuting yourself. Participate!
\item Say thank you\ldots{}by unmuting yourself or following up with the alumni/speaker in an email.
\end{enumerate}
\end{document}