% Created 2021-02-02 Tue 08:19
% Intended LaTeX compiler: pdflatex
\documentclass[11pt]{article}
\usepackage[utf8]{inputenc}
\usepackage{lmodern}
\usepackage[T1]{fontenc}
\usepackage{fixltx2e}
\usepackage{graphicx}
\usepackage{longtable}
\usepackage{float}
\usepackage{wrapfig}
\usepackage{rotating}
\usepackage[normalem]{ulem}
\usepackage{amsmath}
\usepackage{textcomp}
\usepackage{marvosym}
\usepackage{wasysym}
\usepackage{amssymb}
\usepackage{amsmath}
\usepackage[theorems, skins]{tcolorbox}
\usepackage[version=3]{mhchem}
\usepackage[numbers,super,sort&compress]{natbib}
\usepackage{natmove}
\usepackage{url}
\usepackage{minted}
\usepackage[strings]{underscore}
\usepackage[linktocpage,pdfstartview=FitH,colorlinks,
linkcolor=blue,anchorcolor=blue,
citecolor=blue,filecolor=blue,menucolor=blue,urlcolor=blue]{hyperref}
\usepackage{attachfile}
\usepackage[left=1in, right=1in, top=1in, bottom=1in, nohead]{geometry}
\geometry{margin=1.0in}
\usepackage{amsmath}
\usepackage{graphicx}
\usepackage{epstopdf}
\usepackage{fancyhdr}
\usepackage{hyperref}
\usepackage[labelfont=bf]{caption}
\usepackage{setspace}
\def\dbar{{\mathchar'26\mkern-12mu d}}
\pagestyle{fancy}
\fancyhf{}
\renewcommand{\headrulewidth}{0.5pt}
\renewcommand{\footrulewidth}{0.5pt}
\lfoot{\today}
\cfoot{\copyright\ 2021 W.\ F.\ Schneider}
\rfoot{\thepage}
\title{University of Notre Dame\\Physical Chemistry for Chemical Engineers\\(CHE 30324)}
\author{Prof. William F.\ Schneider}
\def\dbar{{\mathchar'26\mkern-12mu d}}
\usepackage[small]{titlesec}
\titlespacing*{\section}
{0pt}{0.4\baselineskip}{0.0\baselineskip}
\titlespacing*{\subsection}
{0pt}{0.4\baselineskip}{0.0\baselineskip}
\titlespacing*{\subsubsection}
{0pt}{0.1\baselineskip}{0.0\baselineskip}
\setcounter{secnumdepth}{3}
\author{William F. Schneider}
\date{\today}
\title{CBE 20290  Syllabus}
\begin{document}

\begin{OPTIONS}
\end{OPTIONS}

\begin{center}
\textsc{Career Choices for Chemical Engineers (CBE 20290)}

University of Notre Dame, Spring 2021
\end{center}

\begin{center}
\begin{tabular}{lr}
\hline
Prof.~Bill Schneider & \href{https://www.nd.edu/\~wschnei1}{web}\\
wschneider@nd.edu, phone 574-631-8754 & \href{https://www.linkedin.com/in/william-schneider-570091a/}{LinkedIn}\\
\href{https://notredame.zoom.us/j/91572218330?pwd=WFFvRW9DU3UvMHhXUTBwQUNIZzd0dz09}{Zoom link} & Meeting times F 4:05-4:55\\
\hline
\end{tabular}
\end{center}

\section{Your Career Choices}
\label{sec:org989ffc2}
As you have undoubtedly already figured out, Chemical Engineering is a demanding major. Your courses will prepare you to solve complex technical problems, to work in teams, and to communicate effectively.  Those skills can take your career in many different directions. This course is designed to expose you to some of those many options and to provide some of the professional skills needed to pursue them. Specifically, the learning goals of this 1 credit course are:

\begin{enumerate}
\item to be exposed to chemical engineering career options, through interactions with alumni who have preceeded you in this journey
\item to begin to discern options that are appealling to you
\item to develop  professional networking and presentation skills that will help get you where you want to go
\end{enumerate}

\section{Format}
\label{sec:org654d2e4}
The course will be delivered virtually. We will use the same \href{https://notredame.zoom.us/j/91572218330?pwd=WFFvRW9DU3UvMHhXUTBwQUNIZzd0dz09}{Zoom link} throughout the semester. If it is helpful, you can consult this \href{https://calendar.google.com/calendar/u/0?cid=Y183NG02cDJnYWQ2NDQ4OTUzZGthaHJia2Nnc0Bncm91cC5jYWxlbmRhci5nb29nbGUuY29t}{Google calendar}.  Weeks 1-3 will focus on professional skill development. Weeks 4-14 will involve a series of virtual panel discussions with alumni of the department. Each panel will focus on one sector, such as pharmaceuticals, energy, foods, personal care, \ldots.  A group of students will introduce the topic and interview the visitors.
\section{Schedule}
\label{sec:org631310a}
\begin{center}
\begin{tabular}{lll}
February 5 & Introduction/Resumes/Handshake & \href{https://www.linkedin.com/in/jared-mrozinske-3b815532/}{Jared Mrozinske}, Career Center\\
February 12 & Interviewing & \href{https://www.linkedin.com/in/jared-mrozinske-3b815532/}{Jared Mrozinske}, Career Center\\
February 19 & LinkedIn/Networking & \href{https://www.linkedin.com/in/melaniesanchezjones/}{Melanie Sanchez}\\
March 5 & Alumni Panel: Consumer/personal goods & student interviewer team\\
March 12 & Alumni Panel: Biotechnology & student interviewer team\\
March 19 & Alumni Panel: Energy \& Renewables & student interviewer team\\
March 26 & Alumni Panel: Oil \& Gas & student interviewer team\\
April 2 & Good Friday & \\
April 9 & Alumni Panel: Chemicals & student interviewer team\\
April 16 & Alumni Panel: Foods & student interviewer team\\
April 23 & Alumni Panel: Professional/Grad school & student interviewer team\\
April 30 & Alumni Panel: Information technology & student interviewer team\\
May 7 & Alumni Panel: Entrepeneurship & student interviewer team\\
May 14 & wrap-up & \\
\end{tabular}
\end{center}

\section{Assignments}
\label{sec:org07dd1d3}
Course materials and assignments will be posted at  \url{https://github.com/wmfschneider/CBE20290}.

\section{Grading}
\label{sec:org96315be}
The course is graded S/U. To receive an S, you must complete all assignments.

\section{Mental Health}
\label{sec:org442ebbe}
Care and Wellness Consultants provide support and resources to students who are experiencing stressful or difficult situations that may be interfering with academic progress. Through Care and Wellness Consultants, students can be referred to The University Counseling
Center (for cost-free and confidential psychological and psychiatric services from
licensed professionals), University Health Services (which provides primary care,
psychiatric services, case management, and a pharmacy), and The McDonald
Center for Student Well Being (for problems with sleep, stress, and substance
use). Visit \url{https://care.nd.edu}.

\section{Professional courtesy}
\label{sec:orgd65fecd}
\begin{enumerate}
\item Arrive at each zoom session on time,  early if possible.
\item Turn your camera on. Make sure you are sitting up and acting professionally.
\item Dress appropriately.
\item Research the organization, the presenters’ backgrounds, or industries. Be prepared!
\item You can ask questions through the chat or by unmuting yourself. Participate!
\item Say thank you\ldots{}by unmuting yourself or following up with the alumni/speaker in an email.
\end{enumerate}
\end{document}